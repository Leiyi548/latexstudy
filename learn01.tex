\documentclass[UTF8]{ctexart}
\usepackage{graphicx}  %引入图片的包
\begin{document}
\textbf{这是个粗体设置}单独一个回车是空格,两个回车是换行
\textit{这是个斜体}
\underline{这是个下划线}

\section{这是第一个章节}
我们可以在这个章节下面来添加自己的文字

\subsection{这是一个子章节}
子章节下面的内容

\subsubsection{这是个三级章节}
三级章节下面的内容

\section{这是第二个章节}
随便添加一些内容

\begin{equation} %将公式写在单独一行
    a^2+b^2=c^2
\end{equation}
%这是加图片的绝对路径导入
\includegraphics[width=0.5\textwidth]{/Users/macos/Desktop/计算机操作系统01.pdf} 

\begin{enumerate}
    \item  列表项1
    \item  列表项2
    \item  列表项3
    \item  列表项4
\end{enumerate}

爱因斯坦在1905年发现的质能守恒方程为: $E=mc^2$

%分数 用%分隔
\begin{equation}
    d={k \varphi(n)+1} \over e
\end{equation}


\begin{tabular}{ c c c}
    单元格1 & 单元格2 & 单元格3 \\
    单元格4 & 单元格5 & 单元格6 \\
    单元格7 & 单元格8 & 单元格9 \\
\end{tabular}


\begin{table}
    \center
    %添加边框竖着的边框
    \begin{tabular}{ |c |c |c|}
        \hline %添加水平方向的边框
        \hline %双横线效果
        单元格1 & 单元格2 & 单元格3 \\
        \hline
        单元格4 & 单元格5 & 单元格6 \\
        \hline
        单元格7 & 单元格8 & 单元格9 \\
        \hline
    \end{tabular}
    \caption{你可以在这里输入表格的标题}
\end{table}

\end{document}